%Fill these in and they'll propagate across the title page. Remember to keep the space at the end!
\def \ajankohta {Tammikuu 2014 }
\def \ajankohtaenglish {August 2021 }
\def \authorname {Nguyen To }
\def \thesistitle {Sign Language Recognition }
\def \thesissubtext {Inflated 3D ConvNet  }
\def \campus {Joensuu }
\def \facultyschooleng {School of Computer Science }
\def \facultyschoolfin { }
%FT = PhD, FM = MSc
\def \supervisorsfin { }
\def \supervisorseng {Profressor Xiao-Zhi Gao }

\def \documenttypeeng {Master's Thesis }
\def \documenttypefin { }
%\def \documenttypeeng {Bachelor's Thesis}
%\def \thesistypefin {Kandidaatintutkielma}

%Number of non-cover pages, to last page of references
\def \mypagecount {42 }
\def \myappendixcount {1 }
\def \myappendixpagecount {32 }

\graphicspath{ {./images/} }
%%%%%%%%%%%%%%%%%% TITLE PAGE %%%%%%%%%%%%%%%%%%

\vspace*{3cm}
\vspace{0.5cm}

\begin{center}
\begin{LARGE}\thesistitle \end{LARGE}

\begin{Large}\thesissubtext \end{Large} 

\vspace{1.5cm}

\begin{Large}\authorname \end{Large}

\vspace{\stretch{1}}

{\large
\documenttypeeng
~\\
% to have it in black and white, swap the commented line
\includegraphics[width=7cm]{{UEF logo.png}}\\
% \includegraphics[width=7cm]{UEF_fin_pysty_1_black}\\
Faculty of Science and Forestry\\
\facultyschooleng \\
}
\end{center}

\vspace{0.5cm}

\thispagestyle{empty}

\begin{spacing}{1.0}
\newpage

%%%%%%%%%%%%%%%%%% Abstract page in English %%%%%%%%%%%%%%%%%%

UNIVERSITY OF EASTERN FINLAND, Faculty of Science and Forestry, \campus School of Computing\\
\facultyschooleng \\ \\
Student, \authorname : \thesistitle \\
\documenttypeeng , \mypagecount p.\\
Supervisors of the \documenttypeeng : \supervisorseng \\
\ajankohtaenglish \\


Abstract:

Effective communication is considered as the foremost fundamental of human skills. However, more than 5\% of the world's population is suffering from disabling hearing loss, as indicated by the World Health Organization (WHO). There is hence a communication gap between the hearing-impaired community, whose primary means of communication is sign language, and others who are not privy with this language. To this end, Sign Language Recognition could be an essential instrument which utilizing vision-based technology and helps the hearing-impaired communicate with the society with ease, thereby diminishing the verbal exchange barrier. The first step in interpreting and analyzing communication via gestures is word-level sign language recognition (WSLR). Recognizing signs from recordings maybe a tough challenge due to the fact that the meaning of a word is determined by a combination of subtle body movements, hand gestures, and other actions. Be that as it may, with the significant advancement of technology, notably Convolutional Neural Network (CNN), in this paper, an Inflated 3D Networks (I3D), a 3D video categorization solution are used to method as an answer of WSLR.


% Key words English

% Edellisellä sivulla olevien suomenkielisten avainsanojen käännökset

~\\ % Tämä tekee tyhjän rivin; älä editoi tätä pois
Keywords:
Sign Language, Continuous Sign Language, Sign Language Recognition, Classification,
Vidieo Classification, Recognition

% CR-luokat

% ACM-luokitus löytyy Computing Reviews -lehden jokaisen
% vuosikerran ensimmäisestä numerosta sekä verkosta
% osoitteesta http://www.acm.org/class/

% Ota omat luokkasi tuoreimmasta vuosikerrasta.

CR Categories (ACM Computing Classification System,
1998 version): A.m, K.3.2\\

\end{spacing}

\newpage


%%%%%%%%%%%%%%%%%% Foreword/Preface %%%%%%%%%%%%%%%%%%

\section*{Preface}
The basis for the project initially originated from my ardor for developing better methods of sign language recognition. The hearing-impaired community's life has changed considerably over the past half-century as a result of policy adjustments and latest technological tendencies, yet the road to find a viable answer for word-level sign language recognition (WSLR) keeps on being a drawn out circumstance. It is my passion to not solely find out, however to develop tools to bridge the communication gap between the deaf community and the society. This project follows the reference and citation guidelines of the "Quo Valdis, Action Recognition? A New Model and the Kinetics Datasets” by a group of Jo\~{a}o Carreira and Andrew Zisserman.

In truth, I could not have achieved my current level of success without a strong support group. To begin with, I wish to express my sincere thanks to my supervisor, Profressor Xiao-Zhi Gao, for his excellent guidance, valuable input and support throughout the entire period.
Furthermore, I would also like to thank Li Dongxu for his enormously valued assistance in collecting data for this study.
Especially with respect Cong Phan, a Phd candidate at Griffith University who gave a great help by offering several useful insights and recommendations. 
And finally, I am grateful to Mai Khanh Nguyen Ngoc. She stood by my side and provided me with the support I needed to complete this thesis. 
\newpage

%%%%%%%%%%%%%%%%%% Abbrieviations %%%%%%%%%%%%%%%%%%

\section*{List of Abbreviations}

\begin{tabular}{lp{12.5cm}}

ACM & Association for Computing Machinery \\

ISY & Itä-Suomen yliopisto \\

UEF & University of Eastern Finland\\

WSLR & Word-level Sign Language Recognition\\

I3D & Two -Stream Inflated 3D Convolutional Network\\

CNN & Convolutional Neural Network\\

LSTM & Long-short Term Memory\\

TGCN & Temporal Graph Convolutional Network\\

\end{tabular}

\newpage


% ----------------- Table of Contents -------------

% Älä tee seuraavaan mitään muutoksia:

\setlength{\parskip}{0ex}

\tableofcontents
\newpage

\listoftables
\newpage

\listoffigures
\newpage

\setlength{\parskip}{2ex}