\documentclass[a4paper, 12pt]{article}
% Allow the usage of graphics (.png, .jpg)
\usepackage[pdftex]{graphicx}
\graphicspath{ {./images/} }
\usepackage{apacite}
\usepackage[T1]{fontenc}
\usepackage{url}
\usepackage{amsmath}
\usepackage{algorithm}
\usepackage{algpseudocode}

\makeatletter
\def\BState{\State\hskip-\ALG@thistlm}
\renewcommand{\ALG@name}{Compute Distance}
\renewcommand{\thealgorithm}{}
\makeatother

% set line spacing
\usepackage{setspace}
\setlength{\parindent}{0pt}
\setlength{\parskip}{2ex}
\linespread{1.3}

% caption
\usepackage{caption} 
\captionsetup[table]{skip=10pt}

\usepackage{tocloft} % for list of equations
% define list of equations
\newcommand{\listequationsname}{\Large{List of Equations}}
\newlistof{myequations}{equ}{\listequationsname}
\newcommand{\myequations}[1]{
   \addcontentsline{equ}{myequations}{\protect\numberline{\theequation}#1}
}
\setlength{\cftmyequationsnumwidth}{2.3em}
\setlength{\cftmyequationsindent}{1.5em}

% Comment the following line to NOT allow the usage of umlauts
\usepackage[utf8]{inputenc}
%custom margins
\usepackage[]{vmargin}
\setpapersize{A4}	
\setmarginsrb{35mm}{30mm}{30mm}{20mm}{0pt}{0mm}{12pt}{13mm}
% Correct hyphenation in urls
\usepackage{url}
% Support long tables
\usepackage[]{longtable}
\usepackage{ltablex}
\usepackage{adjustbox}
%Pretty bibliography in UEF format
\usepackage{natbib}
% verbatim code listings
\usepackage{listings}
%Unobtrusive in document links
\usepackage{hyperref}
\hypersetup{
    colorlinks,
    citecolor=black,
    filecolor=black,
    linkcolor=black,
    urlcolor=black
}
%%%%%%%%%%%%%%%%%%%%%%%%%%UEF FORMAT%%%%%%%%%%%%%%%%%%%%%%%%%%%%%%%%%


\begin{document}

% do cover pages
%Fill these in and they'll propagate across the title page. Remember to keep the space at the end!
\def \ajankohta {Tammikuu 2014 }
\def \ajankohtaenglish {August 2021 }
\def \authorname {Nguyen To }
\def \thesistitle {Sign Language Recognition }
\def \thesissubtext {Inflated 3D ConvNet  }
\def \campus {Joensuu }
\def \facultyschooleng {School of Computer Science }
\def \facultyschoolfin { }
%FT = PhD, FM = MSc
\def \supervisorsfin { }
\def \supervisorseng {Profressor Xiao-Zhi Gao }

\def \documenttypeeng {Master's Thesis }
\def \documenttypefin { }
%\def \documenttypeeng {Bachelor's Thesis}
%\def \thesistypefin {Kandidaatintutkielma}

%Number of non-cover pages, to last page of references
\def \mypagecount {42 }
\def \myappendixcount {1 }
\def \myappendixpagecount {32 }

\graphicspath{ {./images/} }
%%%%%%%%%%%%%%%%%% TITLE PAGE %%%%%%%%%%%%%%%%%%

\vspace*{3cm}
\vspace{0.5cm}

\begin{center}
\begin{LARGE}\thesistitle \end{LARGE}

\begin{Large}\thesissubtext \end{Large} 

\vspace{1.5cm}

\begin{Large}\authorname \end{Large}

\vspace{\stretch{1}}

{\large
\documenttypeeng
~\\
% to have it in black and white, swap the commented line
\includegraphics[width=7cm]{{UEF logo.png}}\\
% \includegraphics[width=7cm]{UEF_fin_pysty_1_black}\\
Faculty of Science and Forestry\\
\facultyschooleng \\
}
\end{center}

\vspace{0.5cm}

\thispagestyle{empty}

\begin{spacing}{1.0}
\newpage

%%%%%%%%%%%%%%%%%% Abstract page in English %%%%%%%%%%%%%%%%%%

UNIVERSITY OF EASTERN FINLAND, Faculty of Science and Forestry, \campus School of Computing\\
\facultyschooleng \\ \\
Student, \authorname : \thesistitle \\
\documenttypeeng , \mypagecount p.\\
Supervisors of the \documenttypeeng : \supervisorseng \\
\ajankohtaenglish \\


Abstract:

Effective communication is considered as the foremost fundamental of human skills. However, more than 5\% of the world's population is suffering from disabling hearing loss, as indicated by the World Health Organization (WHO). There is hence a communication gap between the hearing-impaired community, whose primary means of communication is sign language, and others who are not privy with this language. To this end, Sign Language Recognition could be an essential instrument which utilizing vision-based technology and helps the hearing-impaired communicate with the society with ease, thereby diminishing the verbal exchange barrier. The first step in interpreting and analyzing communication via gestures is word-level sign language recognition (WSLR). Recognizing signs from recordings maybe a tough challenge due to the fact that the meaning of a word is determined by a combination of subtle body movements, hand gestures, and other actions. Be that as it may, with the significant advancement of technology, notably Convolutional Neural Network (CNN), in this paper, an Inflated 3D Networks (I3D), a 3D video categorization solution are used to method as an answer of WSLR.


% Key words English

% Edellisellä sivulla olevien suomenkielisten avainsanojen käännökset

~\\ % Tämä tekee tyhjän rivin; älä editoi tätä pois
Keywords:
Sign Language, Continuous Sign Language, Sign Language Recognition, Classification,
Vidieo Classification, Recognition

% CR-luokat

% ACM-luokitus löytyy Computing Reviews -lehden jokaisen
% vuosikerran ensimmäisestä numerosta sekä verkosta
% osoitteesta http://www.acm.org/class/

% Ota omat luokkasi tuoreimmasta vuosikerrasta.

CR Categories (ACM Computing Classification System,
1998 version): A.m, K.3.2\\

\end{spacing}

\newpage


%%%%%%%%%%%%%%%%%% Foreword/Preface %%%%%%%%%%%%%%%%%%

\section*{Preface}
The basis for the project initially originated from my ardor for developing better methods of sign language recognition. The hearing-impaired community's life has changed considerably over the past half-century as a result of policy adjustments and latest technological tendencies, yet the road to find a viable answer for word-level sign language recognition (WSLR) keeps on being a drawn out circumstance. It is my passion to not solely find out, however to develop tools to bridge the communication gap between the deaf community and the society. This project follows the reference and citation guidelines of the "Quo Valdis, Action Recognition? A New Model and the Kinetics Datasets” by a group of Jo\~{a}o Carreira and Andrew Zisserman.

In truth, I could not have achieved my current level of success without a strong support group. To begin with, I wish to express my sincere thanks to my supervisor, Profressor Xiao-Zhi Gao, for his excellent guidance, valuable input and support throughout the entire period.
Furthermore, I would also like to thank Li Dongxu for his enormously valued assistance in collecting data for this study.
Especially with respect Cong Phan, a Phd candidate at Griffith University who gave a great help by offering several useful insights and recommendations. 
And finally, I am grateful to Mai Khanh Nguyen Ngoc. She stood by my side and provided me with the support I needed to complete this thesis. 
\newpage

%%%%%%%%%%%%%%%%%% Abbrieviations %%%%%%%%%%%%%%%%%%

\section*{List of Abbreviations}

\begin{tabular}{lp{12.5cm}}

ACM & Association for Computing Machinery \\

ISY & Itä-Suomen yliopisto \\

UEF & University of Eastern Finland\\

WSLR & Word-level Sign Language Recognition\\

I3D & Two -Stream Inflated 3D Convolutional Network\\

CNN & Convolutional Neural Network\\

LSTM & Long-short Term Memory\\

TGCN & Temporal Graph Convolutional Network\\

\end{tabular}

\newpage


% ----------------- Table of Contents -------------

% Älä tee seuraavaan mitään muutoksia:

\setlength{\parskip}{0ex}

\tableofcontents
\newpage

\listoftables
\newpage

\listoffigures
\newpage

\setlength{\parskip}{2ex}

\pagenumbering{arabic} % arabic(1234) numbering, autoreset to 1


\section{Introduction}
Sign Language, any methods of communicating by bodily motions, particularly with hands and arms, that is utilized when verbal communication is either difficult or undesirable. Sign language can consist of a series of overly-exaggerated facial expressions, shrugs, or hand gestures; or it can be a fine and delicate mix of hand signals that are complemented by facial expressions and words spelt out using a manual alphabet. When a deaf person or someone speaking a different language is communicating with someone who is hearing, using sign language can help connect the parties. \citep{signlanguagedefinition2020}. The public has neither the time nor the patience to learn sign language, which is complicated and time-consuming to learn and practice. Additionally, there are also many language and culture-specific \citep{holtz2014reading} (e.g Germany, Japanese) constraints which will hinder the widespread adoption of sign language.
Significant advances in deep learning (DL) and improvements in device capabilities, such as computation power, memory capacity, power usage, sensor resolution, and optics, have improved the performance and cost-effectiveness of vision-based applications, allowing them to spread more quickly in the market place.
For this reason, it is interesting to examine sign language recognition (SLR), which automatically translates sign language and aids deaf-mute individuals in communicating with others in their life

Back to the history of 90s, Yann LeCun et al. published "Gradient-Based Learning Applied to Document Recognition", which is widely considered to be the most popular AI article from the era. This paper was the first modern application of convolutional neural networks to be developed.
Since then, more and more sophisticated models trained on ever-larger datasets have been built using the convenient approach of convolutional neural networks. Especially in the field of Computer Vision - Human-based activity recognition, there are many methods can be applied to solve the problem, from traditional convolutional neural networks such as CNN-RNN, CNN-LSTM to RestNetCRNN, Conv3D and state-of-the-art networks e.g Pose-TGCN, I3D.
Inheriting the idea of using two-stream I3D network, which is based 2D ConvNet \citep{carreira2017quo}, presented by Carreira and Zisserman, this project is re-implement the model with a slightly modification inside. It might not be better when comparing with other models, however during the project, I have got many experience and broaden my knowledge on the field of Deep Learning. 

\subsection{Problem}
As same as other human-based activity recognition, SRL also shares some common problems such as background clutter, lightning or lightning changing in a video, motion blur, angle of camera, changing scale. 
SRL, on the other hand, is a more difficult task than ordinary action recognition. Firstly, sign language relies on a combination of global body movement and subtle hand/arm gesture. Additionally, depending on how many times they are repeated, same gestures might have different meanings. SRL might be more difficult to examine because of different states of motions and signers such as localism, gesture speed, preferred hand or physical form. Finally, it is also expensive to collect additional data from many signers even though it is desirable \citep{jiang2021skeleton}.

As described above, the datasets that uses for training SLR are limited, even the number of samples inside each dataset. The table below describes some datasets that normally use for researching.

\begin{table}[ht!]
    \centering
    \def\arraystretch{1}%
    \caption{Sign Language Datasets.}
    \resizebox{\textwidth}{!}{
        \begin{tabular}{|l|l|l|l|l|l|}
            \hline \textbf{Dataset} & \textbf{Language} & \textbf{Classes} & \textbf{Samples} & \textbf{Data Type} & \textbf{Language Level} \\
            \hline CSL Dataset I & Chinese & 500 & 125,000 & Video \& Depth from Kinect & Isolated \\
            \hline CSL Dataset II & Chinese & 100 & 25,000 & Videos \& Depth from Kinect & Continuous \\
            \hline RWTH-PHOENIX-Weather 2014 & German & 1,081 & 6,841 & Videos & Continuous \\
            \hline RWTH-PHOENIX-Weather 2014 T & German & 1,066 & 8,257 & Videos & Continuous \\
            \hline ASLLVD & American & 3,300 & 9,800 & Videos(multiple angles) & Isolated \\
            \hline ASLLVD-Skeleton & American & 3,300 & 9,800 & Skeleton & Isolated \\
            \hline SIGNUM & German & 450 & 33,210 & Videos & Continuous \\
            \hline DGS Kinect 40 & German & 40 & 3,000 & Videos(multiple angles) & Isolated \\
            \hline DEVISIGN-G & Chinese & 36 & 432 & Videos & Isolated \\
            \hline DEVISIGN-D & Chinese & 500 & 6,000 & Videos & Isolated \\
            \hline DEVISIGN-L & Chinese & 2000 & 24,000 & Videos & Isolated \\
            \hline LSA64 & Argentinian & 64 & 3,200 & Videos & Isolated \\
            \hline GSL isol. & Greek & 310 & 40,785 & Videos \& Depth from RealSense & Isolated \\
            \hline GSL SD & Greek & 310 & 10,290 & Videos \& Depth from RealSense & Continuous \\
            \hline GSL SI & Greek & 310 & 10,290 & Videos \& Depth from RealSense & Continuous \\
            \hline IIITA -ROBITA & Indian & 23 & 605 & Videos & Isolated \\
            \hline PSL Kinect & Polish & 30 & 300 & Videos \& Depth from Kinect & Isolated \\
            \hline PSL ToF & Polish & 84 & 1,680 & Videos \& Depth from ToF camera & Isolated \\
            \hline BUHMAP-DB & Turkish & 8 & 440 & Videos & Isolated \\
            \hline LSE-Sign & Spanish & 2,400 & 2,400 & Videos & Isolated \\
            \hline Purdue RVL-SLLL & American & 39 & 546 & Videos & Isolated \\
            \hline RWTH-BOSTON-50 & American & 50 & 483 & Videos(multiple angles) & Isolated \\
            \hline RWTH-BOSTON-104 & American & 104 & 201 & Videos(multiple angles) & Continuous \\
            \hline RWTH-BOSTON-400 & American & 400 & 843 & Videos & Continuous \\
            \hline WLASL & American & 2,000 & 21,083 & Videos & Isolated \\
            \hline
        \end{tabular}
    }
    \label{table:1}
\end{table}

\section{Making a Bibliography}
Copy and paste the following into Verbosus' bibliography page, then make citations, like John Smith did \cite{smith99} under instruction of Jane Doe \cite{doe95}

\begin{lstlisting}[frame=single, columns=fullflexible, basicstyle=\ttfamily]
@article{doe95,
 author="Jane Doe",
 title="Another Article",
 journal="Journal of Journaling",
 year="1995"
}

@book{smith99,
 author="John Smith",
 title="Writing nicely",
 publisher="Incredible Books Inc",
 year="1999"
}
\end{lstlisting}


\newpage

% Two common styles, pick one or find alternate bst.
% \citet{john} is like John (1985); \citep{jane} is like (Jane 1985)
%\bibliographystyle{IEEEtranN} % Like this (Jane 1995)
%\bibliographystyle{plain} % Like this [6] 
\bibliographystyle{apacite}
\bibliography{document}

\end{document}
